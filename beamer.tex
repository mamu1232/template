\documentclass[12pt, unicode]{beamer} %handout をoptionに入れればpauseなどを無視した配布用のPDFが作れる
\usepackage{luatexja}
\usepackage[ipaex]{luatexja-preset}
\usepackage{amsmath,amssymb,mathtools,bm,physics,exscale}
\usepackage{siunitx}
\renewcommand{\kanjifamilydefault}{\gtdefault}%  既定をゴシック体に
\usetheme{Copenhagen}

\setbeamertemplate{navigation symbols}{} % 下にあるナビゲーションシンボルを消す
\setbeamertemplate{footline}[frame number]
\setbeamerfont{footline}{size=\small}
\setbeamertemplate{headline}{}
\usefonttheme{professionalfonts}

%section ごとに目次挿入
\AtBeginSection[]{
    \begin{frame}{Next}
        \tableofcontents[currentsection]
    \end{frame}
}

% Babel (日本語の場合のみ・英語の場合は不要)
\uselanguage{japanese}
\languagepath{japanese}
\deftranslation[to=japanese]{Theorem}{定理}
\deftranslation[to=japanese]{Lemma}{補題}
\deftranslation[to=japanese]{Example}{例}
\deftranslation[to=japanese]{Examples}{例}
\deftranslation[to=japanese]{Definition}{定義}
\deftranslation[to=japanese]{Definitions}{定義}
\deftranslation[to=japanese]{Problem}{問題}
\deftranslation[to=japanese]{Solution}{解}
\deftranslation[to=japanese]{Fact}{事実}
\deftranslation[to=japanese]{Proof}{証明}
\def\proofname{証明}


\title{Beamer Tutorial}
\author[今村]{今村 陽}
\date[\today]{\today}
\institute[京大]{京都大学} % 所属
% [..]に省略名が書ける

\begin{document}

\frame{\maketitle}

\begin{frame}
  \frametitle{Contents}
  \tableofcontents %目次
\end{frame}

\section{ブロック}
\begin{frame}{frame title}
\begin{block}{ブロック}
 これはBlockの中身です.
\end{block}
\end{frame}

\section{pause}
\begin{frame}
 \frametitle{はじめてのスライド}
 \begin{definition}
   1と自分自身しか約数を持たない数を\alert{素数}という.
 \end{definition}
   \begin{itemize}
     \item 2 は素数.
     \pause
     \item 3 も素数.
     \pause
     \item 4 は素数ではない.
  \end{itemize}
 \end{frame}

\section{Overlay}
\begin{frame}{素数は無限に存在する}
\begin{theorem}
素数は無限に存在する.
\end{theorem}
\begin{proof}
 \begin{enumerate}
  \item<1-> $p_1, \dots, p_n$ がすべての素数だと仮定する.% <1>,<2-4>なども可
  \item<2-> $q := p_1 \dots p_n + 1$ とおく.
  \item<3-> すると $q$ はどの $p_1,\dots,p_n$ でも割り切れない.
  \item<1-> したがって,$q$は新たな素数を約数にもつことになって矛盾する.\qedhere
\end{enumerate}
\end{proof}
\uncover<4->{証明は\alert{背理法}を使った.} % テキストにOverlayをつける
\end{frame}

\begin{frame}{columnsを使用した配置}
  \begin{block}{top block}
    ...
  \end{block}
  \begin{columns}[c]  % 中央をあわせる
  %\begin{columns}[t] % 上辺をあわせる
    \begin{column}{0.3\textwidth} % 横幅の30%
      \includegraphics[width=\columnwidth]{I-V.pdf}
    \end{column}
    \begin{column}{0.65\textwidth} % 横幅の65%
      \begin{block}{right block}
        \begin{itemize}
          \item 図の説明とか
        \end{itemize}
      \end{block}
    \end{column}
  \end{columns}
  \begin{block}{bottom block}
    ...
  \end{block}
\end{frame}


\end{document}
